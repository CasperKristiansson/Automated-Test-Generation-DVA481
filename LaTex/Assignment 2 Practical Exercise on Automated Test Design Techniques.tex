\documentclass{article}
\usepackage{listings}
\usepackage{color}
\usepackage{xcolor}

\usepackage{cite}
\usepackage{tabularx}
\usepackage{graphicx} % Required for inserting images
\usepackage{dirtytalk}
\usepackage{pgfplotstable} 
\usepackage{pgfplots}
\usepackage{datatool}
\usepackage{siunitx}
\usepackage[hyphens]{url}  % Allows line breaks at hyphens
\usepackage{hyperref}
\usepackage{graphicx}
\usepackage{microtype}
\usepackage{float}


\hypersetup{
    colorlinks=true,
    linkcolor=blue,
    filecolor=blue,      
    urlcolor=blue,
    citecolor=blue,
}

\definecolor{codegreen}{rgb}{0,0.6,0}
\definecolor{codegray}{rgb}{0.5,0.5,0.5}
\definecolor{codepurple}{rgb}{0.58,0,0.82}
\definecolor{backcolour}{rgb}{0.95,0.95,0.92} % Definition of 'backcolour'

\lstdefinestyle{mystyle}{
    backgroundcolor=\color{backcolour},   
    commentstyle=\color{codegreen},
    keywordstyle=\color{magenta},
    numberstyle=\tiny\color{codegray},
    stringstyle=\color{codepurple},
    basicstyle=\ttfamily\footnotesize,
    breakatwhitespace=false,         
    breaklines=true,                 
    captionpos=b,                    
    keepspaces=true,                 
    numbers=left,                    
    numbersep=5pt,                  
    showspaces=false,                
    showstringspaces=false,
    showtabs=false,                  
    tabsize=2
}

\lstset{style=mystyle}


\title{Assignment 2\\Practical Exercise on Automated Test Design Techniques}
\author{Casper Kristiansson}
\date{\today}

\begin{document}

\maketitle

\section{Taxonomy}
I choose to continue with the automated test generation tool Pynguin. I think the specific details in the taxonomy are still sufficient and don't need any modifications.

\begin{enumerate}
    \item \textbf{Software Artifact}
    \begin{enumerate}
        \item \textbf{Implementation}: Source, generates test based on python source code.
        \item \textbf{Characteristics}: Application/System because it generates tests for Python applications/systems
        \item \textbf{Programming Language}: Python
    \end{enumerate}
    \item \textbf{Implementation}
    \begin{enumerate}
        \item \textbf{Code}: Method level because the testing tools generate unit tests for different methods
        \item \textbf{Monitoring}: Dynamic, because it executes the code and then monitors its behavior
    \end{enumerate}
    \item \textbf{Test Generation}
    \begin{enumerate}
        \item \textbf{Objective}: Code coverage, \say{...aims to automatically generate unit tests that maximize code coverage.} \cite{lukasczyk2020automated}.
        \item \textbf{Technology}: Search-Based Testing, Chapter 3 \cite{lukasczyk2020automated}.
    \end{enumerate}
    \item \textbf{Test Execution}
    \begin{enumerate}
        \item \textbf{Online/Offline}: Offline
    \end{enumerate}
    \item \textbf{Test Oracle}
    \begin{enumerate}
        \item \textbf{Categories}: Manual, \say{So far, Pynguin focuses on test-input generation and excludes the generation of oracles.} \cite{lukasczyk2020automated}.
    \end{enumerate}
\end{enumerate}

\section{Example Program}
I chose to create a simple class in Python which represents a simple bank system. The class is able to create multiple accounts, get balance, withdraw money, and lastly deposit money. In each function, I handle simple error cases such as the account doesn't exist or insufficient balance during withdrawal. \\

\begin{lstlisting}[language=Python]
class BankingSystem:
    def __init__(self):
        self.accounts = {}

    def create_account(self, account_number, name, initial_balance):
        if account_number in self.accounts:
            return "Account already exists"
        if initial_balance < 0:
            return "Initial balance must be non-negative"
        self.accounts[account_number] = {
            "name": name,
            "balance": initial_balance
        }
        return "Account created successfully"

    def deposit(self, account_number, amount):
        if account_number not in self.accounts:
            return "Account not found"
        if amount <= 0:
            return "Deposit amount must be positive"
        self.accounts[account_number]["balance"] += amount
        return "Deposit successful"

    def withdraw(self, account_number, amount):
        if account_number not in self.accounts:
            return "Account not found"
        if amount <= 0:
            return "Withdrawal amount must be positive"
        if amount > self.accounts[account_number]["balance"]:
            return "Insufficient funds"
        self.accounts[account_number]["balance"] -= amount
        return "Withdrawal successful"

    def get_balance(self, account_number):
        if account_number not in self.accounts:
            return "Account not found"
        return self.accounts[account_number]["balance"]
\end{lstlisting}

\section{Automated Generated Test Cases}
When using the tool Pynguin tool to automatically generate tests we get some interesting test cases generated. The tool generated a total of 13 test cases. Each test case tests the expected behaviour and after running a coverage test on the test cases it got a score of 100\%. While it is not mentioned in the assignment I want to quickly note the problem with the generated test cases. For the test case 9 (test\_case\_9) the test tries to withdraw money. A good testing behaviour for this would be to withdraw and than check that the new balance has the correct amount. But rather than withdrawing a proper amount (ex 100) it withdraws using True. While this works in python (adding True to a int results in True being 1) it doesn't really test the true behaviour of the class. Here are a few note worthy test cases generated: \\


\begin{lstlisting}[language=Python]
@pytest.mark.xfail(strict=True)
def test_case_0():
    banking_system_0 = module_0.BankingSystem()
    banking_system_0.create_account(
        banking_system_0, banking_system_0, banking_system_0
    )

def test_case_1():
    banking_system_0 = module_0.BankingSystem()
    var_0 = banking_system_0.deposit(banking_system_0, banking_system_0)
    assert var_0 == "Account not found"

def test_case_5():
    bool_0 = False
    banking_system_0 = module_0.BankingSystem()
    var_0 = banking_system_0.create_account(bool_0, bool_0, bool_0)
    assert var_0 == "Account created successfully"
    assert banking_system_0.accounts == {False: {"name": False, "balance": False}}

@pytest.mark.xfail(strict=True)
def test_case_6():
    none_type_0 = None
    int_0 = -425
    banking_system_0 = module_0.BankingSystem()
    var_0 = banking_system_0.create_account(int_0, int_0, int_0)
    assert var_0 == "Initial balance must be non-negative"
    var_0.withdraw(none_type_0, none_type_0)

@pytest.mark.xfail(strict=True)
def test_case_9():
    bool_0 = True
    banking_system_0 = module_0.BankingSystem()
    var_0 = banking_system_0.create_account(banking_system_0, bool_0, bool_0)
    assert var_0 == "Account created successfully"
    assert len(banking_system_0.accounts) == 1
    var_1 = banking_system_0.withdraw(banking_system_0, bool_0)
    assert var_1 == "Withdrawal successful"
    module_1.object(*var_0)
\end{lstlisting}

\hspace{0cm}
\newpage

\bibliographystyle{IEEEtran}
\bibliography{main}

\end{document}